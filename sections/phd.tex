\section{Case Study}
% ------------------------------------------------------------------------------------------
\begin{frame}{Rate dependece of pCVD diamond pad detectors}
%
  \only<1>{\fig{.5}{pad-full}}
  \only<2>{\fig{.5}{plt}}
%
  \header{Goal:}[\Large]\vspace*{1ex}
  \begin{itemize}
    \item study signal response as a function of incident particle rate
  \end{itemize}\vspace*{2ex}
%
  \header{Main Challenges:}[\Large]\vspace*{1ex}
  \begin{itemize}\itemfill
    \item design and construct experimental setup and run the experiments
    \item analyse the data
  \end{itemize}
%
\end{frame}
% ------------------------------------------------------------------------------------------
\begin{frame}{Rate dependece of pCVD diamond pad detectors}
%
  %
  \begin{minipage}{.6\textwidth}
    \begin{itemize}\itemspace{2ex}
      \item part of the CERN RD42 collaboration (\num{\sim100} members)
      \item detectors developed and provided by the collaboration
      \item I was leading group of \numrange{3}{10} during the experiments
    \end{itemize}
  \end{minipage}
  %
  \begin{minipage}{.35\textwidth}\centering
    \Large\bfseries RD42 collaboration\vspace*{-2ex}
    \fig{.6}{rd42}
  \end{minipage}
%
\end{frame}
% ------------------------------------------------------------------------------------------
\begin{bframe}{Silicon \ra diamond}{cms-tracker-m}{1}
%
  \begin{bbox}{.45}
    \item vast majority of tracking detectors have silicon sensors
    \item extensively used in electronics
    \item easy to manipulate electronic properties
    \item abundance in nature
    \item<2> \bad{current detector not radiation tolerant}
  \end{bbox}
%
\end{bframe}
% ------------------------------------------------------------------------------------------
\begin{bframe}{Test site}{hipa}{1.2}\hypersetup{linkcolor=white!70!black}
%
\vspace*{-.3\textheight}
\begin{bbox}{.45}
  \item \ac{HIPA} at \ac{PSI}
  \item \SI{590}{\MeV} bunched proton beam
  \item up to \SI{1.3}{\MW} beam power
%    \item up to \SI{2.2}{\mA} beam current
%    \item bunch spacing \SI{19.75}{\ns}
\end{bbox}
%
\note{
  \begin{itemize}
    \item highest power in the world
  \end{itemize} }
%
\end{bframe}
% ------------------------------------------------------------------------------------------
\begin{bframe}{\pim beam line}{pim1}{1}
%
  \raggedleft
  \begin{bbox}{.42}
    \item secondary pion beam
    \item $\pi^+$ with \SI{260}{\mevc}
    \item particle flux: \SI{1}{\khzcm} up to \SI{20}{\mhzcm}
%    \item beam composition: 97/2/1 ($\pi^+$/$e^+$/$\mu^+$)
  \end{bbox}
%
  \note{
    \begin{itemize}
      \item proton beam shot on graphite target
      \item creating secondary pion beam
    \end{itemize}}
%
\end{bframe}
% ------------------------------------------------------------------------------------------
\begin{frame}{Setup}
%
  \only<1>{\vspace*{.1\textheight}\fig{.4}{setup}\vspace*{.1\textheight}}
  \only<2>{\fig{.6}{tel}}
  \begin{itemize}\itemfill
    \item commissioning beam telescope made from hybrid Si-pixel detectors
    \item reconstruct trajectory of the beam particles
    \item other instruments: digitiser, power supplies, trigger unit, NIM logic \ldots
  \end{itemize}
%
  \only<2>{\note{
    \begin{itemize}
      \item mention Si reference
      \item use either pad or pixel
    \end{itemize} }}
%
\end{frame}% ------------------------------------------------------------------------------------------
\begin{frame}{Setup - challenges \& solutions}\centering
%
  \begin{tabular}[h]{rcl}
    \header{Challenges}[\Large] & $\Longleftrightarrow$ & \header{Solutions}[\Large] \\[4ex]
    \makecell[r]{detector incompatible with\\readout device} & \bullet & \makecell[l]{extending readout software and firmware\\to read analogue levels of the detector}\\[4ex]
    \makecell[r]{problems with reflections and\\signal routing on the PCB} & \bullet & \makecell[l]{investigating chip signals with oscilloscope\\redesigning the PCB with an\\electrical engineer} \\[4ex]
    \makecell[r]{tedious setup and initialising process} & \bullet & \makecell[l]{writing code to automate the process}\\[4ex]
  \end{tabular}
%
\end{frame}
% ------------------------------------------------------------------------------------------
\begin{frame}{Data-acquisition}
%
  \fig{.5}{eudaq}
%
  \begin{itemize}\itemfill
    \item adjusting and further developing existing software
    \item controlling and monitoring several devices
    \item save combined data stream
    \item long process of improvements \ra in the end data was taken autonomously
  \end{itemize}
%
\end{frame}
% ------------------------------------------------------------------------------------------
\begin{frame}{Debugging \ldots}
%
  \fig{.5}{crash}
%
  \begin{itemize}\itemfill
    \item several beam tests \ra each time install setup
    \item many devices, various persons, different programming languages, new stuff \ldots
    \item errors occurred
    \item on-site debugging, programming and developing
  \end{itemize}
%
\end{frame}
% ------------------------------------------------------------------------------------------
\begin{frame}{Analysis}
%
  \begin{subfigures}
    \subfig{.5}{sm-a2-2a-i.jpg}
    \subfig{.5}{sm-b6-1e.jpg}
  \end{subfigures}
%
  \begin{itemize}\itemfill
    \item developed analysis framework from scratch with \num{>10000} lines of code
    \item converting raw data, aligning telescope and data, tracking, analysing
    \item python script utilising several other scripts in python and \cpp
  \end{itemize}
%
\end{frame}
% ------------------------------------------------------------------------------------------
\begin{frame}{Analysis code}
%
  \begin{minipage}{.45\textwidth}
    \begin{itemize}\itemspace{2ex}
      \item modular architecture \ra object-oriented
      \item open-source (\href{https://github.com/diamondIPP/RateAnalysis.git}{github})
      \item well documented
      \item developed \href{https://github.com/michareichmann/rootplots.git}{ROOT wrapper} for plotting data
    \end{itemize}
  \end{minipage}
  %
  \begin{minipage}{.52\textwidth}
    \fig{.8}{uml}
  \end{minipage}
%
\end{frame}
% ------------------------------------------------------------------------------------------
\begin{frame}{Results}\centering
%
  \begin{tbox}[.7]
    \item developed beam telescope
    \item irradiated pCVD diamond depends less than \SI{2}{\percent} on incident particle rate
    \item presented on several international conferences
    \item published in three conference proceedings
    \item \ldots more in my \href{https://doi.org/10.3929/ethz-b-000611857}{thesis}
  \end{tbox}
  %
  \setsansfont{Lato}
  \begin{center}
    \begin{tcolorbox}[colback=red!5!white, colframe=white!70!red, width=.92\textwidth, arc=4mm, halign=center, valign=center]
      \textbf{\textsc{Michael Philipp Reichmann}}
      \tcblower
      \huge\centering
      \textcolor{chapter-color}{\textbf{\textsc{A Particle Tracker for an Extreme Radiation Environment with Strongly Changing Fluxes: pCVD Diamond}}}
    \end{tcolorbox}
  \end{center}
%
\end{frame}
% ------------------------------------------------------------------------------------------